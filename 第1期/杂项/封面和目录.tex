% 五号字体,开明式标点处理,不设置默认字体
\documentclass[UTF8,12pt,punct=kaiming,fontset=none]{ctexart}
\usepackage{fontspec}  % 字体
\usepackage{subcaption}  % 节标题
% 链接和注释
\usepackage[pageanchor,colorlinks=true,linkcolor=black,citecolor=black,urlcolor=black,linktoc=none]{hyperref}
\usepackage{geometry}  % 页面布局
\usepackage{fancyhdr}  % 页眉页脚
\usepackage{graphicx}  % 图片路径
\usepackage[pages=some]{background}  % 封面
\usepackage{bookmark}  % 书签
\usepackage{anyfontsize}  % 字体大小
\usepackage{tocloft}  % 目录
\usepackage{ifthen}  % 判断

% 图片路径
\graphicspath{{figures/}}

% 封面
% \backgroundsetup{
%     scale=1,
%     color=black,
%     opacity=1,
%     angle=0,
%     contents={
%         \includegraphics[width=\paperwidth,height=\paperheight]{封面.png}
%     }
% }

% 字体
\setCJKmainfont{Source Han Serif SC}
\setCJKsansfont{Source Han Sans SC}
\setmainfont{CMU Serif}

% 布局
\geometry{a4paper,left=2cm,right=2cm,top=2.5cm,bottom=2.5cm}
\setlength{\headheight}{25pt}

% 页眉页脚
\pagenumbering{arabic}
\pagestyle{fancy}
\fancyhead[L]{· \hspace{0.1cm} \thepage \hspace{0.1cm} ·}
\fancyhead[C]{红 \hspace{0.08cm} 石 \hspace{0.08cm} 数 \hspace{0.08cm} 电 \hspace{0.08cm} 评 \hspace{0.08cm} 论\\\scriptsize{Review of Redstonic Digital Circuit}}
\fancyhead[R]{第1期\\\scriptsize{2022年1月}}
\fancyhead[C]{红~石~数~电~评~论\\\scriptsize{Review of Redstonic Digital Circuit}}
\fancyhead[R]{2022年1月(第1期)}
\fancyfoot[L,C,R]{}

% 首页页码
\setcounter{page}{1}

\begin{document}

% 封面
\thispagestyle{empty}
\BgThispage
\quad
\newpage

% 标题
\quad
\vspace{-0.3cm}
\begin{center}
    \fontsize{46pt}{54.2pt} \textbf{红石数电评论} \\
    \Large Review of Redstonic Digital Circuit
\end{center}
\vspace{-1cm}

% 目录
\renewcommand{\contentsname}{}
\renewcommand{\cftdot}{\ensuremath{\cdots}}
\renewcommand{\cftsecleader}{\cftdotfill{1.5}}
\renewcommand{\cftsecfont}{\normalfont}
\renewcommand{\cftsecpagefont}{\normalfont}
% \cftsetpnumwidth{5.5cm}
% \cftsetrmarg{5.5cm}

\tableofcontents

% 页码
\newcounter{currentPageNumber}
\setcounter{currentPageNumber}{3}
\newcounter{nextPageNumber}
\newcommand{\addContent}{
    \setcounter{nextPageNumber}{\arabic{currentPageNumber}}
    \ifthenelse{\pageNumber>1}{
        \addtocounter{nextPageNumber}{\pageNumber}
        \addtocounter{nextPageNumber}{-1}
        \addtocontents{toc}{\string\contentsline{section}{\string\hyperlink{page.\arabic{currentPageNumber}}{\fileName}}{\authorName \quad \arabic{currentPageNumber} - \arabic{nextPageNumber}}{}}
    }{
        \addtocontents{toc}{\string\contentsline{section}{\string\hyperlink{page.\arabic{currentPageNumber}}{\fileName}}{\authorName \quad \arabic{currentPageNumber}}{}}
    }
    \bookmark[page={\arabic{currentPageNumber}}]{\fileName}
    \addtocounter{currentPageNumber}{\pageNumber}
}

% TODO: 自动填写模板

% 论文
\addtocontents{toc}{\string\contentsline{part}{论文}{}{}}
\foreach \fileName\authorName\pageNumber in {
    基于基本红石电路的函数绘图显示器/@章鱼千\hspace{-0.015cm}\_zhyq/4,
    基于石墙电路的随机存取存储器/{@辰占鳌头,@NKID00}/2,
    基于石墙控制线的存储器内的微时序特性分析及应用/@NKID00/1,
    基于RV32M标准的运算器实现/{@MorQin,@Nikkeru}/4
}{
    \addContent
}

% 通讯
\addtocontents{toc}{\string\contentsline{part}{通讯}{}{}}
\foreach \fileName\authorName\pageNumber in {
    更高效的乘法器——树状乘法器原理与建造/作者/1,
    串行二进制转十进制方案/作者/1,
    潜影盒存储——不可堆叠物品解码方案/作者/1,
    潜影盒倒序装填器/作者/1
}{
    \addContent
}

\thispagestyle{fancy}

\end{document}
