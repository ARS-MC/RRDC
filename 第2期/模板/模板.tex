% 五号字体,开明式标点处理,不设置默认字体
\documentclass[UTF8,12pt,punct=kaiming,fontset=none]{ctexart}
\usepackage{fontspec}  % 字体
\usepackage{subcaption}  % 节标题
\usepackage[colorlinks=true, linkcolor=magenta, citecolor=magenta, urlcolor=magenta]{hyperref}  % 超链接
\usepackage{geometry}  % 页面布局
\usepackage{fancyhdr}  % 页眉页脚
\usepackage{titlesec}  % 标题
\usepackage{caption}  % 图表标题
\usepackage{floatrow}  % 图表排版
\usepackage{graphicx}  % 图片路径

% 表格(可选)
\usepackage{multirow}
\usepackage{boldline}
% 电路图(可选)
\usepackage{tikz}
\usepackage[european,nooldvoltagedirection]{circuitikz}

% 图片路径
\graphicspath{{figures/}}

% 字体
\setCJKmainfont{Source Han Serif SC}
\setCJKsansfont{Source Han Sans SC}
\setmainfont{CMU Serif}

% 布局
\geometry{a4paper,left=2cm,right=2cm,top=2.5cm,bottom=2.5cm}
\setlength{\headheight}{25pt}

% 图表标题
\DeclareCaptionFont{captionfont}{\small}
\captionsetup{font=captionfont}
\floatsetup{style=plaintop}

% 页眉页脚
\pagenumbering{arabic}
\pagestyle{fancy}
\fancyhead[L]{· \hspace{0.1cm} \thepage \hspace{0.1cm} ·}
\fancyhead[C]{红 \hspace{0.1cm} 石 \hspace{0.1cm} 数 \hspace{0.1cm} 电 \hspace{0.1cm} 评 \hspace{0.1cm} 论\\\scriptsize{Review of Redstonic Digital Circuit}}
\fancyhead[R]{第2期\\\scriptsize{2023年x月}}
\fancyfoot[L,C,R]{}

% 首页页码
\input{页码.inc}

% 标题
\title{\vspace{-1.5cm}标题\vspace{-0.5cm}}
\author{作者\footnote{作者联系方式或背景等}}
\date{}

% 参考文献标注
\newcommand*{\upcite}[1]{
    \textsuperscript{\cite{#1}}
}

\begin{document}
\pdfbookmark{标题}{\thepage}  % 书签
\hypersetup{bookmarksdepth=-1}  % 禁止后续书签
\maketitle
\thispagestyle{fancy}  % 首页页眉页脚
\vspace{-0.7cm}

% 摘要及关键词
\begin{flushright}
    \begin{minipage}[c]{0.91\linewidth}
        \titleformat{\section}[leftmargin]{\sffamily\small\bfseries}{}{0cm}{}
        \titlespacing{\section}{1.5cm}{1ex}{0cm}

        \section{摘 \hspace{0.105cm} 要}
        \small 摘要.

        \section{关键词}
        \small 关键词 \hspace{0.5cm} 关键词
    \end{minipage}
\end{flushright}
\vspace{0.2cm}

% 节标题格式
\titleformat{\section}[hang]{\large\sffamily\bfseries}{\textmd{\thesection}}{0.5cm}{}
\titlespacing{\section}{0cm}{0.5ex}{0.2ex}
\titleformat{\subsection}[hang]{\normalsize\sffamily}{\textmd{\thesubsection}}{0.5cm}{}
\titlespacing{\subsection}{0cm}{0.5ex}{0.2ex}
\setcounter{section}{0}

\section{引言}
当整数$n > 2$时,关于$x, y, z$的不定方程
$$x^n + y^n = z^n$$
无\textbf{正整数(粗体)}解.

\section{节}
\subsection{节}
正文.图表并排如表 \ref{tab:1} 和图 \ref{fig:1} 所示.\upcite{bib:art1}单张图片如图 \ref{fig:a-figure} 所示.

\begin{figure}[H]
    \begin{floatrow}
        \ttabbox
        {
            \caption{表标题}
            \label{tab:1}
        }
        {
            \scalebox{0.8}{\begin{tabular}{c c c}
            \hlineB{3}
            表头 & 表头 & 表头 \\
            \hlineB{3}
            \multirow{2}{*}{\centering 多行单元格} & 单元格 & \multirow{2}{*}{\centering 多行单元格} \\
            \cline{2-2}
            & 单元格 & \\
            \hline
            单元格 & \multicolumn{2}{c}{多列单元格} \\
            \hlineB{3}
            \end{tabular}}
        }
        \ffigbox
        {
            \caption{图标题}
            \label{fig:1}
        }
        {
            \scalebox{0.8}{\begin{circuitikz}[line width=0.8pt]
            \ctikzset{multipoles/thickness=2}
            \ctikzset{chips/scale=1.2}
            \draw (0,0) node[qfpchip, num pins=12, hide numbers, no topmark, external pins width=0](C1){标注};
            \draw (7,0) node[qfpchip, num pins=12, hide numbers, no topmark, external pins width=0](C2){标注};
            \draw (0,-4) node[qfpchip, num pins=12, hide numbers, no topmark, external pins width=0](C3){标注};
            \draw (C1.bpin 4) to[short, -, l=标注] (C3.bpin 12);
            \draw (C1.bpin 6) to[short, -, l=标注] (C3.bpin 10);
            \draw (C1.bpin 8) to[short, -, l=标注] (C2.bpin 2);
            \draw (C3.bpin 8) to[short, -, l=标注] (7,-4) -- (C2.bpin 5);
            \end{circuitikz}}
        }
    \end{floatrow}
\end{figure}

\begin{figure}[H]
    \includegraphics[width=0.3\linewidth]{figure.png}
    \caption{单张图片}
    \label{fig:a-figure}
\end{figure}

多张子图片如图 \ref{fig:a-random-figure} 和图 \ref{fig:another-random-figure} 所示或如图 \ref{fig:some-random-figure} 所示.

\begin{figure}[H]
    \begin{subfigure}{0.3\linewidth}
        \centering
        \includegraphics[width=\linewidth]{figure.png}
        \caption{图片1}
        \label{fig:a-random-figure}
    \end{subfigure}
    \hspace{1cm}
    \begin{subfigure}{0.3\linewidth}
        \centering
        \includegraphics[width=\linewidth]{figure.png}
        \caption{图片2}
        \label{fig:another-random-figure}
    \end{subfigure}
    \caption{两张图片}
    \label{fig:some-random-figure}
\end{figure}

\section{结语}
结语.\upcite{bib:art1}\upcite{bib:misc1}

\bibliographystyle{unsrt}
\bibliography{reference.bib}
\end{document}
